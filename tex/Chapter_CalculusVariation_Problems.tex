%Copyright 2016 R.D. Martin
%This book is free software: you can redistribute it and/or modify it under the terms of the GNU General Public License as published by the Free Software Foundation, either version 3 of the License, or (at your option) any later version.
%
%This book is distributed in the hope that it will be useful, but WITHOUT ANY WARRANTY; without even the implied warranty of MERCHANTABILITY or FITNESS FOR A PARTICULAR PURPOSE.  See the GNU General Public License for more details, http://www.gnu.org/licenses/.
\section{Problems}
\begin{problem}{Highest point on a surface} Find the coordinates of the highest point on the surface given by $z(x,y)=x^2y+xy$ that when projected onto the xy-plane lies on a circle of radius $r=5$ centered at $x=2$ and $y=3$. Use a computer to illustrate the result. You will also likely need to solve the equations numerically.
\label{prob_CalcVar_1}
\end{problem}
\begin{problem}{Box with no lid} Find the dimensions that maximize the volume of a box with no lid, if the total surface area of the box is 1\,m$^2$.
\label{prob_CalcVar_2}
\end{problem}
%\begin{problem}{Hanging Beads} The figure shows 3 beads of equal mass $m$, connected by mass-less inextensible strings of length $l$ over a precipice of length $L$.
%\capfig{0.4\textwidth}{figures/HangingBeads.png}{3 beads of equal mass $m$, connected by mass-less inextensible strings of length $l$ over a precipice of length $L$} 
%
%\textbf{a)} Minimize the potential energy of the system to find the positions ($x_i$, $y_i$) of the 3 beads.
%
%\textbf{b)} Generalize the result from part a) if there are N beads
%
%\textbf{c)} If the beads and strings are replaced by a heavy rope with a mass per unit length of $\mu$, find the differential equation for the curve that describes the shape of the rope and use a computer to plot it.
%\end{problem}

\begin{problem}{Euler-Lagrange equations for integrand that depends on more than one function}\label{prob_CalcVar_3} Show that the stationary value of the functional $S=\int_a^b L(q_1(t),q_2(t),\dots,\dot{q_1}, \dot{q_2},\dots,t)dt$ gives an Euler-Lagrange equation for each $q_i(t)$:
\begin{align*}
\frac{d}{dt}\left(\frac{\partial L}{\partial \dot{q_i}}\right)-\frac{\partial L}{\partial q_i}=0
\end{align*}
\end{problem}

\begin{problem}{Euler-Lagrange equation when there is a second order derivative}\label{prob_CalcVar_4} Show that the equivalent of the Euler-Lagrange equation when the function in the integrand depends on the second derivative of $y$:
\begin{align*}
I=\int_a^b L(y,y',y'',x)dx
\end{align*}
is given by:
\begin{align*}
\frac{d^2}{dx^2}\die{L}{y''}-\frac{d}{dx}\die{L}{y'}+\die{L}{y}=0
\end{align*}
Note that you will have to integrate by parts twice and that the variation of $y'(x)$ is zero at the end points of the integral.
\end{problem}

\begin{problem}{Brachistrochrone}\label{prob_CalcVar_5}Refer to the problem of example \ref{ex:brach}.\\
\textbf{a)} Find the differential equation for the function $y(x)$ that solves the brachistrochrone problem, by minimizing $T$:
\begin{align*}
\sqrt{2g}T=\int_0^{x_b} \frac{\sqrt{1+y'^2}}{\sqrt{y}}dx
\end{align*}
Note that we have set the problem up such that one end of the wire is at the origin, gravity is in the positive $y$ direction, and the constant $2g$ will not change the shape of the wire. Note that a closed form solution in the form $y=f(x)$ is not possible, and only parametric solutions can be obtained (so called ``cycloids'').

\textbf{b)} Repeat part a), with the wire being constrained to have a length of $L$.
\end{problem}

\begin{problem}{Geodesic}\label{prob_CalcVar_6} Determine the functional for the shortest distance between two points on a sphere of radius $a$, and write the differential equation that gives the function $\phi(\theta)$, where $\phi$ and $\theta$ are the azimuthal and polar angles in spherical coordinates, respectively.
\end{problem}

\begin{problem}{Straight line}\label{prob_CalcVar_7} Show that the shortest distance between two points in a plane is a straight line.
\end{problem}

\begin{problem}{Catenary}\label{prob_CalcVar_8} A rope of uniform linear mass density, $\mu$, and total length, $L$, is hung across a precipice of length, $H$ (both sides of the precipice are at the same height). Minimize the potential energy of the rope:
\begin{align*}
V=\int_0^H \mu g y ds
\end{align*} 
to obtain the differential equation for the shape of the rope. Use a computer to solve and plot the resulting curve (choose reasonable values).
\end{problem}

\begin{problem}{Surface and volume of revolution}\label{prob_CalcVar_9}
\textbf{a)} Find the shape of the curve, $y(x)$, passing through two points, $a$ and $b$, that gives the minimal surface of revolution when rotated about the x-axis (see figure). Note that this would be the shape of a soap film formed between two hoops.\\
\textbf{b)} Find the shape of the curve, $y(x)$, passing through two points, $a$ and $b$, that creates a solid of revolution with the minimal moment of inertia about the x-axis (see figure). Assume that the volume is made of a material with uniform density. It will not be possible to get a closed form, write the solution as a differential equation for $y(x)$ or give $x$ in terms of an integral over $y$.
\capfig{0.2\textwidth}{figures/SurfaceOfRevolution.png}{Minimal surface of revolution between two points, problem \ref{prob_CalcVar_9}}\\
\end{problem}

\section{Solutions}

\paragraph{Problem \ref{prob_CalcVar_1}:}
\begin{solution}
\textbf{a)}
 \begin{align*}
2+2=4
\end{align*}
\end{solution}

